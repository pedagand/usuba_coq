\documentclass{beamer}
\usepackage[francais]{babel}
\usepackage[utf8]{inputenc}

\usepackage{hyperref}

% other packages
\usepackage{latexsym,amsmath,xcolor,multicol,booktabs,calligra}
\usepackage{amssymb}
\usepackage{listings}
\usepackage{proof}
\usepackage{color}


% packages and settings for bibtex
\usepackage[backend=bibtex,sorting=none]{biblatex}
\setbeamerfont{footnote}{size=\tiny}   % footnote for bibtex
\setbeamertemplate{bibliography item}[text]   % reference list for bibtex

\newcommand{\Usuba}{\textsc{Usuba}}
\newcommand{\UsubaA}{$\textsc{Usuba}_1$}
\newcommand{\UsubaB}{$\textsc{Usuba}_2$}
\newcommand{\doubleplus}{+\!\!\!+\;}

\author{Samuel \textsc{Vivien}, sous l'encadrement de Pierre-Évariste \textsc{Dagand}}
\title{\Usuba{}}
\date{\today}
\usepackage{ucas}
\include{../rapport/rules}
\definecolor{deepblue}{rgb}{0,0,0.5}
\definecolor{deepred}{rgb}{0.4,0,0}
\definecolor{deeppurple}{rgb}{0.6,0,0.5}
\definecolor{deepgreen}{rgb}{0,0.3,0}
\definecolor{halfgray}{gray}{0.55}



\begin{document}

\begin{frame}
    \titlepage
\end{frame}

\begin{frame}{}
    \tableofcontents
\end{frame}

\section{\Usuba{} aujourd'hui}

\begin{frame}[fragile]
\scriptsize
\begin{lstlisting}
table SubColumn (input:v4) returns (out:v4)
    { 6, 5, 12, 10, 1, 14, 7, 9, 11, 0, 3, 13, 8, 15, 4, 2 }

table SubColumn (input:v4) returns (out:v4)
    { 6, 5, 12, 10, 1, 14, 7, 9, 11, 0, 3, 13, 8, 15, 4, 2 }

node ShiftRows (input:u16[4]) returns (out:u16[4])
vars
let
    out[0] = input[0];        out[1] = input[1] <<< 1;
    out[2] = input[2] <<< 12; out[3] = input[3] <<< 13
tel

node Rectangle (plain:u16[4],key:u16[26][4])
returns (cipher:u16[4])
vars tmp : u16[26][4]
let
    tmp[0] = plain;
    forall i in [0,24] {
        tmp[i+1] = ShiftRows( SubColumn( tmp[i] ^ key[i] ) )
    }
    cipher = tmp[25] ^ key[25]
tel
\end{lstlisting}
\end{frame}

\begin{frame}{Spécificités du langage}
\begin{itemize}
\item<1-> Pas de conditionnelles (\texttt{if} ou \texttt{while})
\item<2-> Pas d'accès mémoire dynamiques
\end{itemize}

\uncover<3->{Cela permet d'avoir un temps d'exécution indépendant des valeurs du calcul}
\end{frame}

\begin{frame}{Limitations du langage}
\begin{itemize}
\item<1-> Pas de système de type
\item<2-> Pas de spécification de la sémantique
\item<3-> La sémantique implémenté est non compositionnelle !
\end{itemize}

\medskip

\onslide<4-6>{On prend $v = [[0, 1], [2, 3], [4, 5]]$}

\onslide<5-6>{Donc $v[0,1][1] = [1, 3]$}

\onslide<6>{On prend $\{x = v[0, 1]; \; y = x[1]\}$, on obtient donc $y = [2, 3]$}

\onslide<7>{Correction : distinguer $v[0,1; \; 1]$ de $v[0, 1][1]$}
\end{frame}

\begin{frame}[fragile]{Amélioration des appels}
\scriptsize
\begin{lstlisting}
node MapRectangle (plain:u16[64][4],key:u16[64][26][4])
returns (cipher:u16[64][4])
vars
let
    forall i in [0,63] {
        chiper[i] = Rectangle( plain[i], key[i] )
    }
tel
\end{lstlisting}

\end{frame}

\begin{frame}[fragile]{Amélioration des appels}
\scriptsize
\begin{lstlisting}
node MapRectangle (plain:u16[64][4],key:u16[64][26][4])
returns (cipher:u16[64][4])
vars
let
    forall i in [0,63] {
        chiper[i] = Rectangle( plain[i], key[i] )
    }
tel
\end{lstlisting}

\begin{lstlisting}
node MapRectangle (plain:u16[64][4],key:u16[64][26][4])
returns (cipher:u16[64][4])
vars
let
    chiper = Rectangle[64]( plain[i], key[i] )
tel
\end{lstlisting}
    
\end{frame}

\section{Un système de type pour \Usuba{}}

\begin{frame}{Grammaire des types}
\begin{figure}[t]
    \onslide<1->{
        \begin{minipage}{0.3\textwidth}
            \ottgrammartabular{
            \ottdir\ottinterrule
            \ottsize\ottinterrule
            \otttypi
            }
        \end{minipage}
    }
    \onslide<2>{
        \begin{minipage}{0.6\textwidth}
            $\textbf{U V}\; 32$ : entier 32 bits classique.

            $\textbf{U H}\; 64$ : entier 64 bits découpé dans 64 registres.

            $u32 = \textbf{U}\; dir\; 32$ et $v4 = \textbf{U}\; dir\; size[4]$.
        \end{minipage}
    }
\end{figure}
\end{frame}
        
\begin{frame}{Grammaire des types}
\begin{figure}[t]
    \onslide<1->{
        \begin{minipage}{0.3\textwidth}
            \ottgrammartabular{
            \ottdir\ottinterrule
            \ottsize\ottinterrule
            \otttypi
            }
        \end{minipage}
    }
    \onslide<1->{
        \begin{minipage}{0.3\textwidth}
        \ottgrammartabular{
            \otttyp\ottinterrule
            \otttypL}
        \end{minipage}
    }
    \onslide<2->{
        \begin{minipage}{0.3\textwidth}
            \ottgrammartabular{
            \otttypc\ottinterrule
            \ottA
            }
        \end{minipage}
    }
\end{figure}
\end{frame}

\begin{frame}{Règles de typage}
\begin{figure}
    \ottusedrule{\ottdruleBinop{}}
    \ottusedrule{\ottdruleTuple{}}
\end{figure}
\end{frame}

\begin{frame}{Règles de typage des appels de nœuds}
    \begin{figure}
        \ottusedrule{\ottdruleFun{}}
    \end{figure}
\end{frame}
    
\begin{frame}{Deux nouvelles constructions}
\begin{itemize}
\item<1-> Les coercions explicites
\item<2-> Les constructeurs de tableaux (pour typer $x + (x[0], x[1])$)
\end{itemize}
\end{frame}

\section{4 sémantiques pour \Usuba{}}

\begin{frame}{Sémantique par évaluation}
\begin{enumerate}
\item<1-> On évalue tout dans l'ordre
\item<2-> Permet de gérer des équations de modifications
\item<3-> Sémantique la plus proche de celle implémenté
\end{enumerate}
\end{frame}

\begin{frame}{Sémantique relationnelle}
\begin{enumerate}
\item<1-> On définie une propriété $e \mapsto v$.
\item<2-> Sémantique indépendante de l'ordre des équations
\item<3-> Sémantique non calculatoire
\item<4-> Accepte beaucoup de systèmes comme $\{y = y\}$
\end{enumerate}
\end{frame}

\begin{frame}{Sémantique par tri topologique}
\begin{enumerate}
\item<1-> Remonte l'ordre des évuations pour calculer les valeurs
\item<2-> Peu maniable pour de la preuve de préservation de la sémantique
\end{enumerate}
\end{frame}

\begin{frame}{Sémantique par point fixe}
\begin{enumerate}
\item<1-> Essaye de calculer les équations par passages successifs sur le système
\item<2-> Sous ensemble stricte de la sémantique relationnelle
\end{enumerate}
\end{frame}

\section{Conclusion}

\begin{frame}{Conclusion}

\end{frame}

\begin{frame}
Merci de m'avoir écouté
\end{frame}

\end{document}